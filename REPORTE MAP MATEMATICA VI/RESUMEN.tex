% set up
\vspace*{\fill}
\begin{center}
    {\fontsize{25pt}{0pt} \MakeUppercase{\textbf{RESUMEN}}}
\vspace*{5pt}
\end{center}
\addcontentsline{toc}{chapter}{RESUMEN}
%-------------------------------------------------------
Como objetivo general del MAP, fue desarrollar y aplicar las SERIES DE FOURIER a funciones periódicas, en este caso para funciones periódicas definida a trozos, hasta un máximo de 3.

En este caso, haciendo uso del lenguaje de computación "Python", aplicando distintos métodos numéricos de integración y operaciones matemáticas básicas, tratando de simular la "SERIE DE FOURIER TRUNCADA". Creando un sistema de comunicación con el usuario "FRONT END".


\vspace*{\fill}
