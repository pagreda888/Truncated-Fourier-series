\chapter{Descripción General}
\label{ch:intro}

\section{Series de Fourier}
\label{ch:concid}
Las series de Fourer describen señales periódicas como la combinación de señales/funciones armónicas. 
El objetivo y la practica de estas series, es describir dichas señales en una suma de funciones trigonométricas de senos y cosenos, con destinas frecuencias con múltiplos de la "frecuencia fundamental".

Serie de Fourier trigonométrica: 
\begin{equation}
    f(t) = a_{0} + \sum_{n=1}^{\infty}\left(a_{n}\cos{\left(nw_{n}t\right)}+b_{n}\sin{\left(nw_{n}t\right)}\right)
    \label{eq:sumFour}
\end{equation}

Coeficientes de Fourier:

\begin{equation}
\begin{split}
    a_{0}=\frac{1}{2L}\int_{-L}^{L} f(t)\, dt \\
    a_{n}=\frac{1}{L}\int_{-L}^{L} f(t)\cos(n\omega_{n}t)\, dt \\
    b_{n}=\frac{1}{L}\int_{-L}^{L} f(t) \sin(n\omega_{n}t)\, dt
\end{split}
\label{eq:coefFour}
\end{equation}

\section{Serie de fourier truncada a 2N + 1 términos:}
\label{ch:concid}
Al querer aproximar una función periódica $f(t)$ la cual es continua por partes y posee infinitos armónicos, tendremos que truncar la función
hasta el armónico N de forma tal que el error sea el mínimo establecido. 

La forma de calcular la energıa de la señal:

\begin{equation}
    E_{f}=\int_{-L}^{L} f^2(t)\, dt
  \label{eq:ex2a}
\end{equation}

\section{Integral cuadrada del error (ICE)}
\label{ch:concid}
El calculo de ICE se utiliza para encontrar el valor N que luego será utilizado para obtener una representación de la Sumatorio de Fourier Truncada.

Serie de Fourier trigonométrica: 
\begin{equation}
    \begin{split}
    ICE(N) = E_{f}- \left\{a^2_{0}T+\frac{T}{2}\sum_{n=1}^{N}\left(a^2_{n}+b^2_{n} \right)\right\}\\
    ICE(N)\leq 0.02E_{f}
    \end{split}
    \label{eq:sumFour}
\end{equation}
%%%%%%%%%%%%%%%%%%%%%%%%%%%%%%%%%%%%%%%%%%%%%%%%%%%%%%%%%%%%%%%%%%%%%%%%%%%%%%%%%%%%%%%%%%%%%%%%%%%%%%%%%%%%%%%%%%%
\chapter{PARTE PRACTICA}
\label{ch:modeloMat}
%%%%%%%%%%%%%%%%%%%%%%%%%%%%%%%%%%%%%%%%%%%%%%%%%%%%%
%%% SUBSECTION NAME
%%%%%%%%%%%%%%%%%%%%%%%%%%%%%%%%%%%%%%%%%%%%%%%%%%%%%
\section{Interfaz de Usuario}
\label{ch:deduccion1}
La interfaz de usuario, se basa en un sistema tipo consola, breve instrucción te uso: 
\begin{itemize}
    \item Ingreso de funciones(Ingreso recursivo):
        \begin{enumerate}
        \item Ingreso de Función/Definición de Función a trozos + ENTER.
        \end{enumerate}
    \item Ingreso de Rango de Operación de Función Ingresada(Ingreso recursivo):
        \begin{enumerate}
        \item Inicio de rango + ENTER.
        \item Fin de rango + ENTER.
        \end{enumerate}
    \item Despliegue de resultados, se podra visualizar .
    \item Al Finalizar en Ingreso de la Función a Trozos, PRESIONAR ENTER, cuando la consola solicite ingreso de nueva función.
    \item Tomar en cuenta las indicaciones de uso en listadas en la interfaz de usuario.
    
\end{itemize}
%(Eq 1. \ref{eq:ex1a}  Eq 1. \ref{eq:ex1b}).
%%%%%%%%%%%%%%%%%%%%%%%%%%%%%%%%%%%% Helicoide central
\subsection{FUNCIÓN A) }

\begin{equation}
    f(t) =
    \begin{cases}
    0 & \text{si } -2 < t < -1 \\
    1 & \text{si } -1 < t < 1 \\
    0 & \text{si } 1 < t < 2
    \end{cases}
\end{equation}

Ingreso de función: 

\begin{figure}[h!]
    \centering
    \includegraphics[width = 0.9\textwidth]{Figuras/Ingreso F1.jpg}
    \caption[Ingreso de la Función A).]%
    {Se puede percibir el ingreso de función y la secuencia a seguir para el ingreso.} 
    \label{fig:img1a}
\end{figure}

Resultados de la Funcion: 

\begin{figure}[h!]
    \centering
    \includegraphics[width = 1\textwidth]{Figuras/Resultados Ingreso F1.jpg}
    \caption[Ingreso de la Función A).]%
    {Se puede percibir el despliegue de resultados para la sumatoria de Fourier para dicha función.} 
    \label{fig:img2a}
\end{figure}

\begin{figure}[h!]
    \centering
    \includegraphics[width = 1\textwidth]{Figuras/Grafica Ingreso F1.jpg}
    \caption[Ingreso de la Función A).]%
    {Se puede percibir la gráfica resultante de la Función A.} 
    \label{fig:img3a}
\end{figure}

\begin{figure}[h!]
    \centering
    \includegraphics[width = 0.9\textwidth]{Figuras/ANS BNS Ingreso F1.jpg}
    \caption[Ingreso de la Función A).]%
    {Resultados de los ANs y BNs para la funcion A)} 
    \label{fig:img4a}
\end{figure}

\hfill
%%%%%%%%%%%%%%%%%%%%%%%%%%%%%%%%%%%% CILINDRO CENTRAL

\subsection{FUNCIÓN B)}
\begin{equation}
    f(t) =
        \begin{cases}
        0 & \text{si } -2 < t < -1 \\
        t + 2 & \text{si } -1 < t < 1 \\
        0 & \text{si } 1 < t < 2
        \end{cases}
\end{equation}

\begin{figure}[h!]
    \centering
    \includegraphics[width = 0.9\textwidth]{Figuras/Ingreso F2.jpg}
    \caption[Ingreso de la Función B).]%
    {Se puede percibir el ingreso de la función B)} 
    \label{fig:img2a}
\end{figure}

\begin{figure}[h!]
    \centering
    \includegraphics[width = 0.9\textwidth]{Figuras/Resultados Ingreso F2.jpg}
    \caption[Ingreso de la Función B).]%
    {Resultados de los ANs y BNs para la función B)} 
    \label{fig:img2b}
\end{figure}

\begin{figure}[h!]
    \centering
    \includegraphics[width = 1\textwidth]{Figuras/Grafica Ingreso F2.jpg}
    \caption[Ingreso de la Función B).]%
    {Se puede percibir la gráfica resultante de la Función B.} 
    \label{fig:img2c}
\end{figure}

\hfill

\chapter{DESARROLLO DE CODIGO}
\label{ch:Resultados}

\section{Función ICE}

\begin{figure}[h!]
    \centering
    \includegraphics[width = 1\textwidth]{Figuras/FUNCION ICE.jpg}
    \caption[EJECUCIÓN.]%
    {Código Python para el algoritmo del calculo de la función ICE.} 
    \label{fig:cod1}
\end{figure}

\section{FUNCIÓN PARA EL INGRESO DEL USUARIO Y REINICIO DE VARIABLES}
\begin{figure}[h!]
    \centering
    \includegraphics[width = 1\textwidth]{Figuras/Input y reinicio.jpg}
    \caption[EJECUCIÓN.]%
    {Menú de ingreso del usuario, y Matrices utilizadas para el almacenamiento de datos.} 
    \label{fig:cod2}
\end{figure}

\section{FUNCIÓN PARA EL CALCULO DE ENERGÍA}
\begin{figure}[h!]
    \centering
    \includegraphics[width = 1\textwidth]{Figuras/Calculo de Energia.jpg}
    \caption[OUTPUT.]%
    {En este caso utilizamos hacemos uso de la matriz de ingreso de la función, para el calculo de la energía.} 
    \label{fig:cod3}
\end{figure}

\section{CALCULO COEFICIENTES DE FOURIER Y FUNCIÓN PARA GRAFICAR}
\begin{figure}[h!]
    \centering
    \includegraphics[width = 1\textwidth]{Figuras/COEFICIENTES DE FOURIER.jpg}
    \caption[OUTPUT.]%
    {Se aprecia el calculo de los coeficientes de Fourier, A0, AN Y BN, al mismo tiempo el desarrollo de la función trigonométrica de Fourier.} 
    \label{fig:cod3}
\end{figure}


\newpage
\chapter{ENLACE DE PROYECTO}
\begin{figure}[h!]
    \centering
    \includegraphics[width = 0.95\textwidth]{Figuras/CODIGOQR.png}
    \caption[ENLACE DE PROYECTO (2)]%
    {ENLACE DE PROYECTO EN EL \href{https://drive.google.com/drive/folders/1jUVlJLyLqmCDVATNPLQzYZXY89hcIUcN?usp=sharing}{LINK}.} 
    \label{fig:QRCODE}
\end{figure}

\paragraph{ACLARACIONES \\}
DENTRO DEL ENLACE ENCUENTRA EL DESARROLLO DEL CÓDIGO DE PROGRAMACIÓN PARA LECTURA Y EJECUCIÓN EN PYTHON, EL CÓDIGO DE LATEX DEL REPORTE Y EL PDF DEL REPORTE. 
